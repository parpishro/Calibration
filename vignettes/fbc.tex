% Options for packages loaded elsewhere
\PassOptionsToPackage{unicode}{hyperref}
\PassOptionsToPackage{hyphens}{url}
%
\documentclass[
]{article}
\usepackage{amsmath,amssymb}
\usepackage{lmodern}
\usepackage{iftex}
\ifPDFTeX
  \usepackage[T1]{fontenc}
  \usepackage[utf8]{inputenc}
  \usepackage{textcomp} % provide euro and other symbols
\else % if luatex or xetex
  \usepackage{unicode-math}
  \defaultfontfeatures{Scale=MatchLowercase}
  \defaultfontfeatures[\rmfamily]{Ligatures=TeX,Scale=1}
\fi
% Use upquote if available, for straight quotes in verbatim environments
\IfFileExists{upquote.sty}{\usepackage{upquote}}{}
\IfFileExists{microtype.sty}{% use microtype if available
  \usepackage[]{microtype}
  \UseMicrotypeSet[protrusion]{basicmath} % disable protrusion for tt fonts
}{}
\makeatletter
\@ifundefined{KOMAClassName}{% if non-KOMA class
  \IfFileExists{parskip.sty}{%
    \usepackage{parskip}
  }{% else
    \setlength{\parindent}{0pt}
    \setlength{\parskip}{6pt plus 2pt minus 1pt}}
}{% if KOMA class
  \KOMAoptions{parskip=half}}
\makeatother
\usepackage{xcolor}
\usepackage[margin=1in]{geometry}
\usepackage{color}
\usepackage{fancyvrb}
\newcommand{\VerbBar}{|}
\newcommand{\VERB}{\Verb[commandchars=\\\{\}]}
\DefineVerbatimEnvironment{Highlighting}{Verbatim}{commandchars=\\\{\}}
% Add ',fontsize=\small' for more characters per line
\usepackage{framed}
\definecolor{shadecolor}{RGB}{248,248,248}
\newenvironment{Shaded}{\begin{snugshade}}{\end{snugshade}}
\newcommand{\AlertTok}[1]{\textcolor[rgb]{0.94,0.16,0.16}{#1}}
\newcommand{\AnnotationTok}[1]{\textcolor[rgb]{0.56,0.35,0.01}{\textbf{\textit{#1}}}}
\newcommand{\AttributeTok}[1]{\textcolor[rgb]{0.77,0.63,0.00}{#1}}
\newcommand{\BaseNTok}[1]{\textcolor[rgb]{0.00,0.00,0.81}{#1}}
\newcommand{\BuiltInTok}[1]{#1}
\newcommand{\CharTok}[1]{\textcolor[rgb]{0.31,0.60,0.02}{#1}}
\newcommand{\CommentTok}[1]{\textcolor[rgb]{0.56,0.35,0.01}{\textit{#1}}}
\newcommand{\CommentVarTok}[1]{\textcolor[rgb]{0.56,0.35,0.01}{\textbf{\textit{#1}}}}
\newcommand{\ConstantTok}[1]{\textcolor[rgb]{0.00,0.00,0.00}{#1}}
\newcommand{\ControlFlowTok}[1]{\textcolor[rgb]{0.13,0.29,0.53}{\textbf{#1}}}
\newcommand{\DataTypeTok}[1]{\textcolor[rgb]{0.13,0.29,0.53}{#1}}
\newcommand{\DecValTok}[1]{\textcolor[rgb]{0.00,0.00,0.81}{#1}}
\newcommand{\DocumentationTok}[1]{\textcolor[rgb]{0.56,0.35,0.01}{\textbf{\textit{#1}}}}
\newcommand{\ErrorTok}[1]{\textcolor[rgb]{0.64,0.00,0.00}{\textbf{#1}}}
\newcommand{\ExtensionTok}[1]{#1}
\newcommand{\FloatTok}[1]{\textcolor[rgb]{0.00,0.00,0.81}{#1}}
\newcommand{\FunctionTok}[1]{\textcolor[rgb]{0.00,0.00,0.00}{#1}}
\newcommand{\ImportTok}[1]{#1}
\newcommand{\InformationTok}[1]{\textcolor[rgb]{0.56,0.35,0.01}{\textbf{\textit{#1}}}}
\newcommand{\KeywordTok}[1]{\textcolor[rgb]{0.13,0.29,0.53}{\textbf{#1}}}
\newcommand{\NormalTok}[1]{#1}
\newcommand{\OperatorTok}[1]{\textcolor[rgb]{0.81,0.36,0.00}{\textbf{#1}}}
\newcommand{\OtherTok}[1]{\textcolor[rgb]{0.56,0.35,0.01}{#1}}
\newcommand{\PreprocessorTok}[1]{\textcolor[rgb]{0.56,0.35,0.01}{\textit{#1}}}
\newcommand{\RegionMarkerTok}[1]{#1}
\newcommand{\SpecialCharTok}[1]{\textcolor[rgb]{0.00,0.00,0.00}{#1}}
\newcommand{\SpecialStringTok}[1]{\textcolor[rgb]{0.31,0.60,0.02}{#1}}
\newcommand{\StringTok}[1]{\textcolor[rgb]{0.31,0.60,0.02}{#1}}
\newcommand{\VariableTok}[1]{\textcolor[rgb]{0.00,0.00,0.00}{#1}}
\newcommand{\VerbatimStringTok}[1]{\textcolor[rgb]{0.31,0.60,0.02}{#1}}
\newcommand{\WarningTok}[1]{\textcolor[rgb]{0.56,0.35,0.01}{\textbf{\textit{#1}}}}
\usepackage{graphicx}
\makeatletter
\def\maxwidth{\ifdim\Gin@nat@width>\linewidth\linewidth\else\Gin@nat@width\fi}
\def\maxheight{\ifdim\Gin@nat@height>\textheight\textheight\else\Gin@nat@height\fi}
\makeatother
% Scale images if necessary, so that they will not overflow the page
% margins by default, and it is still possible to overwrite the defaults
% using explicit options in \includegraphics[width, height, ...]{}
\setkeys{Gin}{width=\maxwidth,height=\maxheight,keepaspectratio}
% Set default figure placement to htbp
\makeatletter
\def\fps@figure{htbp}
\makeatother
\setlength{\emergencystretch}{3em} % prevent overfull lines
\providecommand{\tightlist}{%
  \setlength{\itemsep}{0pt}\setlength{\parskip}{0pt}}
\setcounter{secnumdepth}{-\maxdimen} % remove section numbering
\ifLuaTeX
  \usepackage{selnolig}  % disable illegal ligatures
\fi
\IfFileExists{bookmark.sty}{\usepackage{bookmark}}{\usepackage{hyperref}}
\IfFileExists{xurl.sty}{\usepackage{xurl}}{} % add URL line breaks if available
\urlstyle{same} % disable monospaced font for URLs
\hypersetup{
  pdftitle={FBC: Full Bayesian Calibration},
  pdfauthor={Parham Pishrobat and William J. Welch},
  hidelinks,
  pdfcreator={LaTeX via pandoc}}

\title{FBC: Full Bayesian Calibration}
\author{Parham Pishrobat and William J. Welch}
\date{2023-05-29}

\begin{document}
\maketitle

\hypertarget{introduction}{%
\subsection{Introduction}\label{introduction}}

\texttt{FBC} is a package that uses the data from both physical
experiment and computer simulation to calibrate the simulator model.
Calibration is the process of fitting a statistical model to the
observed data by adjusting calibration parameters. These parameters can
represent various aspects of the simulation model, such as tuning
hyperparameter of the model itself or unknown but fixed physical
properties that are governed by physical system.

The \texttt{FBC} package is a based on the well-known Kennedy and
O'Hagan (KOH) calibration model (2001). In its original formulation, KOH
formulates a hierarchical Bayesian framework with two sequential phases.
In the first phase, model hyperparameters are predicted using maximum
likelihood estimation (MLE) method. In the second phase, Bayesian
analysis is used to derive the posterior distribution of calibration
parameters while fixing the hyperparameters found in the first phase.
Neither KOH's original formulation nor later suggestions are fully
Bayesian as they use MLE methods to estimate hyperparameters, largely
due to computational infeasibility (CITATION: Higdon, Berger, other).

\texttt{FBC} runs a fully Bayesian model that includes both calibration
parameters and model hyperparameters in its Bayesian framework.
Implementation of the package optimizes the calibration process and
memory management to increase computational efficiency. Moreover, the
fully implemented Bayesian framework, enables the user to input the
information about all parameters and hyperparameters in the form of
prior specification. Common prior distribution are implemented in
\texttt{FBC} to allows for high degree of flexibility in specifying the
expert knowledge or the lack thereof. \texttt{FBC} runs a Markov Chain
Monte Carlo (MCMC) algorithm to find the posterior distribution of
calibration parameters and model hyperparameters. Furthermore,
\texttt{FBC} can be used for prediction of mean response and model
discrepancy for a new test input configuration.

(TODO: edit this paragraph to reflect the structure of the vignette in
the end) The current vignette is structured into following sections: The
first section (\texttt{FBC} Usage) explains the package functionality
with a simple pedagogic example. Also in this section, the notation for
inputs and outputs of both computer and physical experiments are
introduced. The second section (Calibration Model) generalizes the
example introduced in the first example to build model components. Using
the general notation while referencing the example, we describe the KOH
calibration model and its modelling choices. The third section
(Parameter Estimation), presents the theoretical results to derive the
posterior distribution of model parameters and MCMC-based estimation of
parameters. In the fourth section (Prediction), provides the results to
derive MCMC-based predictions for new input configuration using the
estimated parameters and MCMC samples of model components. The fifth
section (Implementation) explains some of the implementation features
and choices that distinguishes \texttt{FBC} from other implementations
and provides a reasoning that support those choices.Finally, the last
section (Application) provides two real-world and well-cited examples to
demonstrate the full functionality and limitations of the \texttt{FBC}
package. In the end, an appendix is provided to provide further details
and a reference page to link for further readings.

\pagebreak

\hypertarget{fbc-usage}{%
\subsection{\texorpdfstring{1. \texttt{FBC}
Usage}{1. FBC Usage}}\label{fbc-usage}}

\texttt{FBC} package has two main public functions: \texttt{calibrate()}
and \texttt{predict()}. As the name suggests, \texttt{calibrate()} takes
the field and simulation data to calibrate the simulator model and
\texttt{predict} takes a new input configuration and the calibration
model (\texttt{fbc} object) to predict mean simulator response, model
discrepancy, or mean field response at new input configuration. In
addition, \texttt{FBC} has a few helper function to aid in arguments
entry and visualization.

\hypertarget{setup}{%
\subsubsection{1.1 Setup}\label{setup}}

Building a calibration model requires data from field experiment and
simulation. To focus on the functionality of the package, we use a
simple pedagogic example as experimental setting. Consider an experiment
in which a wiffle ball is dropped from different heights and the time it
takes to hit the ground is measured. This experiment has one
experimental input, height (\(h\)) and one calibration input, gravity
(\(g\)) to produce the response, time (\(t\)). Note that in the field
experiment, the earth`s gravity is fixed but unknown to the experimenter
and therefore it is not part of the input data.

The field experiment with above specification has been performed by
Derek Bingham and Jason Loeppky (CITATION). The data field is loaded in
the package environment under \texttt{ballField} name. To increase
robustness, the response vector \(\bf{t}\) and input matrix \(\bf{h}\)
(a vector in ball example) are packaged into a single input matrix
\texttt{ballField}:

\begin{Shaded}
\begin{Highlighting}[]
\FunctionTok{library}\NormalTok{(FBC)}
\FunctionTok{head}\NormalTok{(ballField, }\DecValTok{3}\NormalTok{)}
\SpecialCharTok{\textgreater{}}\NormalTok{         t     h}
\SpecialCharTok{\textgreater{}}\NormalTok{ [}\DecValTok{1}\NormalTok{,] }\FloatTok{0.27} \FloatTok{0.178}
\SpecialCharTok{\textgreater{}}\NormalTok{ [}\DecValTok{2}\NormalTok{,] }\FloatTok{0.22} \FloatTok{0.356}
\SpecialCharTok{\textgreater{}}\NormalTok{ [}\DecValTok{3}\NormalTok{,] }\FloatTok{0.27} \FloatTok{0.534}
\FunctionTok{dim}\NormalTok{(ballField)}
\SpecialCharTok{\textgreater{}}\NormalTok{ [}\DecValTok{1}\NormalTok{] }\DecValTok{63}  \DecValTok{2}
\end{Highlighting}
\end{Shaded}

Simulation of the ball drop experiment is easy to model using
introductory physics results:

\[
t = \sqrt{\frac{2h}{g}}
\]

We have implemented the above mathematical model as a code that takes
\(h\) and \(g\) in and returns \(t\). The input design matrix is a Latin
Hypercube Design (CITATION) consisting of two columns: h and g. Similar
to field data, the simulation data is also packaged into a single matrix
\texttt{ballSim} by binding the input matrix \([\bf{h} \quad \bf{g}]\)
to resulting vector \(\bf{t}\) so that response is the first column.

\begin{Shaded}
\begin{Highlighting}[]
\FunctionTok{head}\NormalTok{(ballSim, }\DecValTok{3}\NormalTok{)}
\SpecialCharTok{\textgreater{}}\NormalTok{          t     h      g}
\SpecialCharTok{\textgreater{}}\NormalTok{ [}\DecValTok{1}\NormalTok{,] }\FloatTok{0.404} \FloatTok{0.998} \FloatTok{12.220}
\SpecialCharTok{\textgreater{}}\NormalTok{ [}\DecValTok{2}\NormalTok{,] }\FloatTok{0.487} \FloatTok{0.940}  \FloatTok{7.909}
\SpecialCharTok{\textgreater{}}\NormalTok{ [}\DecValTok{3}\NormalTok{,] }\FloatTok{0.450} \FloatTok{0.886}  \FloatTok{8.736}
\FunctionTok{dim}\NormalTok{(ballSim)}
\SpecialCharTok{\textgreater{}}\NormalTok{ [}\DecValTok{1}\NormalTok{] }\DecValTok{100}   \DecValTok{3}
\end{Highlighting}
\end{Shaded}

Figure 1 shows the distribution of time versus height for both physical
and simulation experiments. Note that for higher height values, the
simulation responses (blue) underestimate their corresponding field
response (red), displaying a systemic bias for simulation model.

\pagebreak

\end{document}
